\documentclass[a4paper,oneside,DIV8]{scrartcl}
  
\usepackage[ngerman]{babel}
\usepackage[latin9]{inputenc}
\usepackage{amsmath}
\usepackage{amsthm}
\usepackage{relsize}
\usepackage{color}
\usepackage[a4paper]{geometry} 
\usepackage{pictex}
\usepackage{tikz}
\usepackage{siunitx}
\usepackage{pgf}
\usepackage{mathrsfs}
\usetikzlibrary{arrows}

\makeatletter                             
\def\blfootnote{\xdef\@thefnmark{}\@footnotetext}

 \newcommand{\handouttitle}[4]{
 	\begin{center}
		\Large #4
	\end{center}

	\bigskip
	\noindent
	#1 (\textsf{#2})
	\hfill
	#3

	\noindent
	\rule{\linewidth}{.5pt}
	\bigskip
	\@afterindentfalse\@afterheading}
	
\makeatother
\renewcommand{\sectfont}{\normalfont}

\definecolor{ffqqqq}{rgb}{255,0,0}

\geometry{a4paper, top=25mm, bottom=30mm} 

\begin{document}

\pagestyle{empty}

\handouttitle{D.~Eisele}
	{dominik\_eisele@gmx.de}
	{\today}
	{Newton Verfahren}

\section{Ziel des Newton Verfahrens}
	\begin{itemize}
		\item Ann�herung an die Nullstellen, �ber eine Linearisierung der Funktion an geeigneten Stellen
		\item L�sung von allen nichtlinearen Gleichungen und Gleichungssystemen
		\item L�sung von Gleichungen in der Form: $f(x) = 0$
	\end{itemize}
	
\section{Vorgehen bei Konstruktion}
	\begin{enumerate}
		\item die Tangente der Funktion $f(x)$ im Startpunkt bestimmen
		\item der Schnittpunkt der Tangente mit der x-Achse ist der neue Startpunkt
		\item dieses Verfahren so oft wiederholen bis man die gew�nschte Genauigkeit erreicht hat
	\end{enumerate}
	
	\begin{tikzpicture}[line cap=round,line join=round,>=triangle 45,x=1.0cm,y=0.4cm]
		\draw[->,color=black] (-4.2,0.) -- (4.2,0.);
		
		\foreach \x in {-4,-3,...,-1,1,2,...,4}
		\draw[shift={(\x,0)},color=black] (0pt,2pt) -- (0pt,-2pt) node[below] {\footnotesize $\x$};
		
		\draw[->,color=black] (0,-8.2) -- (0,8.2);
		
		\foreach \y in {-8,-6,...,-2,2,4,6}
		\draw[shift={(0,\y)},color=black] (2pt,0pt) -- (-2pt,0pt) node[left] {\footnotesize $\y$};
				
		\draw[color=black] (0pt,-10pt) node[right] {\footnotesize $0$};
		
		\clip(-4.2,-8.2) rectangle (4.2,8.2);
		
		\draw[color=red,smooth,samples=100,domain=-4.2:4.2] plot(\x,{0.5*(\x)^(3.0)-2.0*(\x)^(2.0)+0.5*(\x)+2.0});
		
		\begin{scriptsize}
			\draw[color=ffqqqq] (2.1,5.2) node {$g(x) = 0,5x^3 - 2x^2 + 0,5x + 2$};
		\end{scriptsize}
		
		\draw[color=green,smooth,samples=100,domain=-4.2:4.2] plot(\x,{10.76624*\x+11.25802});
		
		\draw[color=green,smooth,samples=100,domain=-4.2:4.2] plot(\x,{6.32288*\x+5.33027});
		
		\draw[color=green,smooth,samples=100,domain=-4.2:4.2] plot(\x,{4.93807*\x+4.02045});
		
		\draw[color=green,smooth,samples=100,domain=-4.2:4.2] plot(\x,{4.75103*\x+3.86547});
		
		\draw[color=green,smooth,samples=100,domain=-4.2:4.2] plot(\x,{4.74736*\x+3.86248});
		
		\node (1) at (-1.603,0) {\textsf{X}};
		
		\draw [blue] (-1.603,0) -- (-1.603,-6);

		\node (2) at (-1.60298,-6) {\textsf{X}};
		
		\node (3) at (-1.04568,0) {\textsf{X}};
		
		\draw [blue] (-1.04568,0) -- (-1.04568,-1.21842);
		
		\node (4) at (-1.04568,-1.21842) {\textsf{X}};
		
		\node (5) at (-0.8136,0) {\textsf{X}};
	\end{tikzpicture} 
	
\section{Iterationsformel}
	\begin{equation*}
		\boxed{x_{n+1} = x_{n} - \frac{f(x_{n})}{f^\prime(x_{n})}}
	\end{equation*}

	\begin{enumerate}
		\item Startwert in der N�he der Nullstelle w�hlen
		\item Iteration so lange durchf�hren bis das gew�nschte Ergebnis erzielt ist
	\end{enumerate}

\end{document}