\RequirePackage{etex}
\documentclass{beamer}
\usepackage[latin9]{inputenc}
\usepackage[ngerman]{babel}
\usepackage{amsmath}
\usepackage{listings}
\usepackage{color}
\usepackage{pictex}
\usepackage{tikz}
\usepackage{siunitx}

\title[]{Newton-Verfahren}
\author{Dominik Eisele}
\institute[WSS]{Werner-Siemens-Schule}
\date{\today}
\subject{Newton}
\keywords{Newton, Newton-Verfahren}

\usetheme{Singapore}
\usecolortheme{rose}
\setbeamertemplate{navigation symbols}{}
\setbeamertemplate{footline}[frame number]
\beamersetuncovermixins{\opaqueness<1>{25}}{\opaqueness<2->{15}}

\definecolor{dkgreen}{rgb}{0,0.6,0}
\definecolor{gray}{rgb}{0.5,0.5,0.5}
\definecolor{mauve}{rgb}{0.58,0,0.82}

\lstset{language=C}
\lstset{numbers=left,
	numberstyle=\tiny,
	numbersep=5pt,
	breaklines=true,
	showstringspaces=false,
	frame=l ,
	xleftmargin=15pt,
	xrightmargin=15pt,
	basicstyle=\ttfamily\scriptsize,
	stepnumber=1,
	keywordstyle=\color{blue},		% keyword style
	commentstyle=\color{dkgreen},	% comment style
	stringstyle=\color{mauve}		% string literal style
}

%\AtBeginDocument{\addtobeamertemplate{block begin}{\setlength\abovedisplayskip{0pt}}}
%\AtBeginDocument{\addtobeamertemplate{block begin}{\setlength\abovedisplayshortskip{0pt}}}

\let\oldsqrt\sqrt
\def\sqrt{\mathpalette\DHLhksqrt}
\def\DHLhksqrt#1#2{\setbox0=\hbox{$#1\oldsqrt{#2\,}$}\dimen0=\ht0
\advance\dimen0-0.3\ht0
%0.3 ist das Ma� f�r die Hakenl�nge, relativ zum Inhalt der Wurzel
\setbox2=\hbox{\vrule height\ht0 depth -\dimen0}%
{\box0\lower0.4pt\box2}}

%\include{bild_ganze_folie.tex}

\setcounter{tocdepth}{1}

\begin{document}

\begin{frame}
	\titlepage
\end{frame}

\begin{frame}{Inhalt}
	\tableofcontents
\end{frame}


\section{Allgemeine Informationen}
\subsection{Allgemeine Informationen-dots}

\begin{frame}{Geschichte}
	\begin{itemize}
		\item wird auch Newton-Raphson-Verfahren genannt
		\item benannt nach Sir Isaac Newton (1669) und Joseph Raphson (1690)
		\item ver�ffentlicht 1671 in "`Methodus fluxionum et serierum infinitarum"'
	\end{itemize}
\end{frame}

\begin{frame}{Ziel des Newton-Verfahrens}
	\begin{itemize}
		\item L�sung von nichtlinearen Gleichungen und Gleichungssystemen
		\item Ann�herung an die Nullstellen
		\item L�sung von Gleichungen in der Form: $f(x) = 0$
	\end{itemize}
\end{frame}


\section{Quellen}
\subsection{Quellen-dots}
\begin{frame}{Quellen}
	\begin{itemize}
		\item 
		\item 
		\item
		\item
		\item 
	\end{itemize}
\end{frame}


\end{document}
