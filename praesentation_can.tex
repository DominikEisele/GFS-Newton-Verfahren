\RequirePackage{etex}
\documentclass{beamer}
\usepackage[latin9]{inputenc}
\usepackage[ngerman]{babel}
\usepackage{amsmath}
\usepackage{listings}
\usepackage{color}
\usepackage{pictex}
\usepackage{tikz}
\usepackage{siunitx}
\usepackage{pgf}
\usepackage{mathrsfs}
\usetikzlibrary{arrows}
\usepackage{booktabs}

\title[]{Newton-Verfahren}
\author{Dominik Eisele}
\institute[WSS]{Werner-Siemens-Schule}
\date{\today}
\subject{Newton Verfahren}
\keywords{Newton, Newton-Verfahren}

\usetheme{Singapore}
\usecolortheme{rose}
\setbeamertemplate{navigation symbols}{}
\setbeamertemplate{footline}[frame number]
\beamersetuncovermixins{\opaqueness<1>{25}}{\opaqueness<2->{15}}

\definecolor{dkgreen}{rgb}{0,0.6,0}
\definecolor{gray}{rgb}{0.5,0.5,0.5}
\definecolor{mauve}{rgb}{0.58,0,0.82}
\definecolor{ffqqqq}{rgb}{255,0,0}

\lstset{language=C}
\lstset{numbers=left,
	numberstyle=\tiny,
	numbersep=5pt,
	breaklines=true,
	showstringspaces=false,
	frame=l ,
	xleftmargin=15pt,
	xrightmargin=15pt,
	basicstyle=\ttfamily\scriptsize,
	stepnumber=1,
	keywordstyle=\color{blue},		% keyword style
	commentstyle=\color{dkgreen},		% comment style
	stringstyle=\color{mauve}			% string literal style
}

%\AtBeginDocument{\addtobeamertemplate{block begin}{\setlength\abovedisplayskip{0pt}}}
%\AtBeginDocument{\addtobeamertemplate{block begin}{\setlength\abovedisplayshortskip{0pt}}}

\let\oldsqrt\sqrt
\def\sqrt{\mathpalette\DHLhksqrt}
\def\DHLhksqrt#1#2{\setbox0=\hbox{$#1\oldsqrt{#2\,}$}\dimen0=\ht0
\advance\dimen0-0.3\ht0
%0.3 ist das Ma� f�r die Hakenl�nge, relativ zum Inhalt der Wurzel
\setbox2=\hbox{\vrule height\ht0 depth -\dimen0}%
{\box0\lower0.4pt\box2}}

%\include{bild_ganze_folie.tex}

\setcounter{tocdepth}{1}

\begin{document}

\begin{frame}
	\titlepage
\end{frame}

\begin{frame}{Inhalt}
	\tableofcontents
\end{frame}


\section{Allgemeine Informationen}
\subsection{Allgemeine Informationen-dots}

\begin{frame}{Geschichte}
	\begin{itemize}
		\item wird auch Newton-Raphson-Verfahren genannt
		\item benannt nach Sir Isaac Newton (1669) und Joseph Raphson (1690)
		\item ver�ffentlicht 1671 in "`Methodus fluxionum et serierum infinitarum"'
	\end{itemize}
\end{frame}

\begin{frame}{Ziel des Newton-Verfahrens}
	\begin{itemize}
		\item Ann�herung an die Nullstellen
		\item L�sung von allen nichtlinearen Gleichungen und Gleichungssystemen
		\item L�sung von Gleichungen in der Form: $f(x) = 0$
		\item kommt immer dann zum Einsatz, wenn die Gleichung nicht durch die bekannten Methoden l�sbar ist, wie z.\,B. Mitternachtsformel, quadratische
			Erg�nzung, ...
	\end{itemize}
\end{frame}

\section{Graphische Konstruktion}
\subsection{Graphische Konstruktion-dots}

\begin{frame}{Graphische Konstruktion}
	\begin{tikzpicture}[line cap=round,line join=round,>=triangle 45,x=1.0cm,y=0.4cm]
		\draw[->,color=black] (-4.2,0.) -- (4.2,0.);
		
		\foreach \x in {-4,-3,...,-1,1,2,...,4}
		\draw[shift={(\x,0)},color=black] (0pt,2pt) -- (0pt,-2pt) node[below] {\footnotesize $\x$};
		
		\draw[->,color=black] (0,-8.2) -- (0,8.2);
		
		\foreach \y in {-8,-6,...,-2,2,4,6}
		\draw[shift={(0,\y)},color=black] (2pt,0pt) -- (-2pt,0pt) node[left] {\footnotesize $\y$};
				
		\draw[color=black] (0pt,-10pt) node[right] {\footnotesize $0$};
		
		\clip(-4.2,-8.2) rectangle (4.2,8.2);
		
		\draw[color=red,smooth,samples=100,domain=-4.2:4.2] plot(\x,{0.5*(\x)^(3.0)-2.0*(\x)^(2.0)+0.5*(\x)+2.0});
		
		\begin{scriptsize}
			\draw[color=ffqqqq] (2.1,5.2) node {$g(x) = 0,5x� - 2x� + 0,5x + 2$};
		\end{scriptsize} 																	\pause
		
		\node (1) at (-1.60298,-6) {\textsf{X}};													\pause
		
		\draw[color=green,smooth,samples=100,domain=-4.2:4.2] plot(\x,{10.76624*\x+11.25802});					\pause
		
		\node (2) at (-1.04568,0) {\textsf{X}};														\pause
		
		\draw [blue] (-1.04568,0) -- (-1.04568,-1.21842);	 											\pause
		
		\node (3) at (-1.04568,-1.21842) {\textsf{X}};												\pause
		
		\draw[color=green,smooth,samples=100,domain=-4.2:4.2] plot(\x,{6.32288*\x+5.33027});					\pause
		
		\draw[color=green,smooth,samples=100,domain=-4.2:4.2] plot(\x,{4.93807*\x+4.02045});					\pause
		
		\draw[color=green,smooth,samples=100,domain=-4.2:4.2] plot(\x,{4.75103*\x+3.86547});					\pause
		
		\draw[color=green,smooth,samples=100,domain=-4.2:4.2] plot(\x,{4.74736*\x+3.86248});					\pause
	\end{tikzpicture} 
\end{frame}

\begin{frame}{Funktionswerte Werte}
	\begin{table}[]
		\centering
		
		\begin{tabular}{@{}ll@{}}
			\toprule
			Berechnungsschritt		& Funktionswert \\ \midrule
			0                  			& -1,6029       \\
			1                  			& -1,0456       \\
			2                  			& -0,8430       \\
			3                  			& -0,8141       \\
			4                  			& -0,8136       \\
			5                  			& -0,8136       \\
			Berechneter Wert   		& -0,8136       \\ \bottomrule
		\end{tabular}
		\end{table}
\end{frame}

\section{Quellen}
\subsection{Quellen-dots}
\begin{frame}{Quellen}
	\begin{itemize}
		\item 
		\item 
		\item
		\item
		\item 
	\end{itemize}
\end{frame}


\end{document}
